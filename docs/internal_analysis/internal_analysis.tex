\documentclass[12pt,a4paper]{article}
\usepackage[utf8]{inputenc}
\usepackage{amsmath}
\usepackage{amssymb}
\usepackage{graphicx}
\usepackage{tikz}
\usepackage{booktabs}
\usepackage{float}
\usepackage{hyperref}
\usepackage{enumitem}
\usepackage[margin=2.5cm]{geometry}
\usepackage{titlesec}
\usepackage{xcolor}
\usepackage{mdframed}
\usepackage{longtable}
\usepackage{placeins}
\usepackage{cleveref}
\usepackage{algorithm}
\usepackage{algpseudocode}
\usepackage{tikz}
\usetikzlibrary{shapes,arrows,positioning,shapes.geometric}

% Define modern colors
\definecolor{definitionheader}{RGB}{41, 128, 185} % Blue
\definecolor{explanationheader}{RGB}{46, 204, 113} % Green
\definecolor{observationheader}{RGB}{230, 126, 34} % Orange
\definecolor{boxbg}{RGB}{248, 249, 250} % Light gray background

% Style configurations
\titleformat{\section}
{\normalfont\LARGE\bfseries}{\thesection}{1em}{}[\vspace{0.2cm}]
\titleformat{\subsection}
{\normalfont\Large\bfseries}{\thesubsection}{1em}{}[\vspace{0.1cm}]

% Modern box styles
\mdfdefinestyle{definitionstyle}{
    linecolor=definitionheader,
    backgroundcolor=boxbg,
    linewidth=0.5pt,
    roundcorner=5pt,
    innertopmargin=10pt,
    innerbottommargin=10pt,
    innerleftmargin=12pt,
    innerrightmargin=12pt,
    skipabove=\baselineskip,
    skipbelow=\baselineskip,
    needspace=3\baselineskip,
    frametitlebackgroundcolor=definitionheader,
    frametitlefont=\normalfont\bfseries\color{white},
    frametitlerule=true,
    frametitleaboveskip=3pt,
    frametitlebelowskip=7pt,
    frametitlerulewidth=0pt
}

\mdfdefinestyle{explanationstyle}{
    linecolor=explanationheader,
    backgroundcolor=boxbg,
    linewidth=0.5pt,
    roundcorner=5pt,
    innertopmargin=10pt,
    innerbottommargin=10pt,
    innerleftmargin=12pt,
    innerrightmargin=12pt,
    skipabove=\baselineskip,
    skipbelow=\baselineskip,
    needspace=3\baselineskip,
    frametitlebackgroundcolor=explanationheader,
    frametitlefont=\normalfont\bfseries\color{white},
    frametitlerule=true,
    frametitleaboveskip=3pt,
    frametitlebelowskip=7pt,
    frametitlerulewidth=0pt
}

\mdfdefinestyle{observationstyle}{
    linecolor=observationheader,
    backgroundcolor=boxbg,
    linewidth=0.5pt,
    roundcorner=5pt,
    innertopmargin=10pt,
    innerbottommargin=10pt,
    innerleftmargin=12pt,
    innerrightmargin=12pt,
    skipabove=\baselineskip,
    skipbelow=\baselineskip,
    needspace=3\baselineskip,
    frametitlebackgroundcolor=observationheader,
    frametitlefont=\normalfont\bfseries\color{white},
    frametitlerule=true,
    frametitleaboveskip=3pt,
    frametitlebelowskip=7pt,
    frametitlerulewidth=0pt
}

% Custom environments with modern headers
\newenvironment{definition}[1]
{\begin{mdframed}[style=definitionstyle,frametitle={Definition: #1}]}
{\end{mdframed}}

\newenvironment{explanation}
{\begin{mdframed}[style=explanationstyle,frametitle={Explanation}]}
{\end{mdframed}}

\newenvironment{observation}
{\begin{mdframed}[style=observationstyle,frametitle={Observation}]}
{\end{mdframed}}

% Hyperref configuration
\hypersetup{
    colorlinks=true,
    linkcolor=black,
    filecolor=black,
    urlcolor=blue,
    citecolor=black
}

% List styling
\setlist[itemize]{leftmargin=*,label=\textbullet,itemsep=0.2em,parsep=0.2em}

\title{\textbf{\LARGE Internal Analysis \& Target-Based Planning}}
\author{\Large}
\date{\today}

\begin{document}

\maketitle
\thispagestyle{empty}
\tableofcontents
\clearpage

\section{Part I: Internal Analysis}

\subsection{Overview}

\begin{explanation}
Internal Analysis is designed to help organizations understand and optimize their team interactions, dependencies, and workflow efficiency. This analysis focuses on three key aspects:
\begin{itemize}
    \item Team dependencies and their impact
    \item Work distribution and flow efficiency
    \item Coordination costs and optimization opportunities
\end{itemize}
\end{explanation}

\section{Team Dependency Analysis}

\subsection{Motivation}

\begin{explanation}
Team dependencies in modern organizations can significantly impact productivity, lead times, and costs. Understanding these dependencies helps:
\begin{itemize}
    \item Identify bottlenecks and inefficiencies
    \item Optimize team structures and interactions
    \item Reduce coordination overhead
    \item Improve delivery predictability
\end{itemize}
\end{explanation}

\subsection{Key Metrics and Formulations}

\subsubsection{Dependency Impact Score (DIS)}

\begin{definition}{Dependency Impact Score}
\[
DIS = \sum(W_i \times D_i \times C_i)
\]
Where:
\begin{itemize}
    \item $W_i$ = Work item volume between teams i
    \item $D_i$ = Dependency strength (1-5 scale)
    \item $C_i$ = Coordination cost factor
\end{itemize}
\end{definition}

\begin{observation}
Dependency Strength Scale:
\begin{itemize}
    \item 1: Minimal - Occasional information sharing
    \item 2: Low - Regular updates needed
    \item 3: Medium - Shared deliverables
    \item 4: High - Critical path dependencies
    \item 5: Critical - Blocking dependencies
\end{itemize}
\end{observation}

\subsubsection{Flow Efficiency Analysis}

\begin{definition}{Flow Efficiency}
\[
FE = \frac{\sum VAT}{\sum LT} \times 100\%
\]
Where:
\begin{itemize}
    \item VAT = Value-added time (actual work)
    \item LT = Total lead time (including wait times)
\end{itemize}
\end{definition}

\begin{explanation}
This metric helps understand:
\begin{itemize}
    \item Actual value-adding activities vs. wait times
    \item Process efficiency
    \item Opportunity areas for improvement
\end{itemize}
\end{explanation}

\subsection{Input Parameters}

\begin{definition}{Input Parameters}
Key parameters for analysis:
\begin{itemize}
    \item Team Size: Number of team members (Integer)
    \item Dependencies: Team dependency mapping (Matrix)
    \item WIP Limits: Maximum concurrent work (Integer)
    \item Throughput: Items completed per time (Float)
\end{itemize}

Example Dependency Matrix:
\[
\begin{matrix}
    & T1 & T2 & T3 & T4 \\
T1 & - & 3 & 1 & 0 \\
T2 & 3 & - & 2 & 1 \\
T3 & 1 & 2 & - & 4 \\
T4 & 0 & 1 & 4 & -
\end{matrix}
\]
\end{definition}

\subsection{System Performance Indicators}

\begin{definition}{Performance Metrics}
\begin{itemize}
    \item Flow Efficiency: $FE = \frac{VAT}{LT}$ (Target: > 40\%)
    \item WIP Impact: $WIS = \sum(\frac{WIP_i}{T_i})$ (Target: < 1.5)
    \item Lead Time: $SLT = \sum(LT_i \times \frac{D_i}{D_{max}})$ (Target: Varies)
\end{itemize}
\end{definition}

\section{Distribution Models}

\subsection{Even Distribution Model}

\begin{definition}{Dependency Balance Index}
\[
DBI = \frac{\sigma(D_i)}{\mu(D_i)}
\]
Target: DBI < 0.3 indicates good balance
\end{definition}

\begin{explanation}
Key Characteristics:
\begin{itemize}
    \item Equal number of dependencies per team
    \item Balanced workload distribution
    \item Minimized coordination overhead
\end{itemize}
\end{explanation}

\subsection{Uneven Distribution (Hub-and-Spoke)}

\begin{definition}{Centrality Impact}
\[
CI = \left(\frac{D_c}{D_{avg}}\right) \times \left(\frac{T_c}{T_{avg}}\right)
\]
Where:
\begin{itemize}
    \item $D_c$ = Central team dependencies
    \item $D_{avg}$ = Average team dependencies
    \item $T_c$ = Central team throughput
    \item $T_{avg}$ = Average team throughput
\end{itemize}
\end{definition}

\subsection{Cost Analysis}

\begin{definition}{Cost Structure}
Total Cost = Direct Costs + Indirect Costs + Overhead

Where:
\begin{itemize}
    \item Direct Costs = Meeting costs + Communication time
    \item Indirect Costs = Context switching + Wait times
    \item Overhead = Management coordination + Tools
\end{itemize}

Annual Cost per Team:
\[
AC = BC \times (1 + DF)
\]
Where:
\begin{itemize}
    \item BC = Team Size × Hourly Rate × Annual Hours
    \item DF = Number of Dependencies × 0.15
\end{itemize}
\end{definition}

\section{Part II: Target-Based Planning}

\begin{explanation}
Target-Based Planning simplifies platform solution development by working backwards:
\begin{itemize}
    \item Start with the desired business outcome (e.g., "reduce ticket processing time by 30\%")
    \item Define clear, measurable targets
    \item Design the minimum viable platform features needed to achieve these targets
\end{itemize}
This approach helps avoid over-engineering by focusing only on platform capabilities that directly contribute to target achievement.
\end{explanation}

\begin{observation}
Two main approaches for platform solutions:
\begin{itemize}
    \item Team-Based Platform: Focuses on team collaboration and workflow automation
    \item Ticket-Based Platform: Emphasizes process automation and routing optimization
\end{itemize}
The choice depends on which metrics you want to improve most: team efficiency or process throughput.
\end{observation}

\subsection{Model-Specific Analysis}

\subsubsection{Team-Based Planning}

\begin{explanation}
Team-based planning focuses on human resource optimization. Instead of starting with detailed team structures and workflows, it begins with target outcomes:
\begin{itemize}
    \item Desired cost savings or efficiency improvements
    \item Required service level maintenance or improvements
    \item Acceptable timeline for changes
\end{itemize}
This approach helps avoid over-engineering team structures and focuses on necessary changes to meet business goals.
\end{explanation}

\begin{definition}{Maximum Allowable Investment (Team)}
\[
MAI_{team} = \min(0.5 \times AC_{team}, 1.5 \times TS)
\]
Where:
\begin{itemize}
    \item $AC_{team}$ = Annual labor cost (team-based)
    \item $TS$ = Target savings over ROI period
\end{itemize}
\end{definition}

\begin{observation}
The MAI formula enforces two key constraints:
\begin{itemize}
    \item Investment cannot exceed 50\% of current annual costs (practicality constraint)
    \item Investment must be justified by 150\% of target savings (ROI constraint)
\end{itemize}
These constraints help ensure that proposed changes remain practical and economically viable.
\end{observation}

\begin{definition}{Required Team Efficiency Gains}
\[
RTEG = \min\left(\frac{MS}{N \times H \times W}, MC\right)
\]
Where:
\begin{itemize}
    \item $MS$ = Monthly savings target
    \item $N$ = Team size
    \item $H$ = Hourly rate
    \item $W$ = Working hours
    \item $MC$ = Current manual work percentage
\end{itemize}
\end{definition}

\begin{explanation}
The RTEG formula helps determine if targets are achievable:
\begin{itemize}
    \item It calculates the minimum efficiency improvement needed
    \item Caps improvements at current manual work percentage (can't automate more than exists)
    \item Provides early warning if targets are unrealistic
\end{itemize}
This helps teams quickly identify if targets need adjustment before detailed planning begins.
\end{explanation}

\subsubsection{Ticket-Based Planning}

\begin{explanation}
Ticket-based planning focuses on process optimization. It starts with desired outcomes in terms of:
\begin{itemize}
    \item Target processing times
    \item Quality improvements
    \item Cost per ticket reduction
\end{itemize}
This approach is particularly effective for service desk, customer support, and other ticket-driven operations.
\end{explanation}

\begin{definition}{Maximum Allowable Investment (Ticket)}
\[
MAI_{ticket} = \min(0.5 \times AC_{ticket}, 1.5 \times TS)
\]
Where:
\begin{itemize}
    \item $AC_{ticket}$ = Annual cost (ticket volume × cost per ticket)
    \item $TS$ = Target savings over ROI period
\end{itemize}
\end{definition}

\begin{observation}
The ticket-based MAI follows similar constraints as team-based but considers:
\begin{itemize}
    \item Volume scalability (higher volumes may justify higher investment)
    \item Process standardization potential
    \item Automation opportunity size
\end{itemize}
This helps ensure investments scale appropriately with ticket volumes and complexity.
\end{observation}

\begin{definition}{Required Ticket Efficiency Gains}
\[
RTEG = \min\left(\frac{MS}{M \times T \times P}, PE\right)
\]
Where:
\begin{itemize}
    \item $MS$ = Monthly savings target
    \item $M$ = Monthly tickets
    \item $T$ = Hours per ticket
    \item $P$ = People per ticket
    \item $PE$ = Current processing efficiency
\end{itemize}
\end{definition}

\begin{explanation}
The ticket efficiency formula helps organizations:
\begin{itemize}
    \item Identify minimum required improvement per ticket
    \item Account for current process inefficiencies
    \item Consider both time and resource optimization
\end{itemize}
This granular approach helps teams focus on specific aspects of ticket processing that need improvement.
\end{explanation}

\section{Break-even Analysis}

\begin{explanation}
Break-even analysis in target-based planning differs from traditional approaches:
\begin{itemize}
    \item Starts with required break-even period
    \item Works backwards to determine required monthly improvements
    \item Considers different scenarios to achieve the same target
\end{itemize}
This helps teams explore multiple paths to the same goal while maintaining focus on the target timeline.
\end{explanation}

\subsection{Team-Based Scenarios}

\begin{explanation}
Scenarios in target-based planning provide structured paths to achieve the desired outcomes. Each scenario represents a different risk-reward balance:
\begin{itemize}
    \item Conservative: Minimal disruption, longer timeline
    \item Moderate: Balanced approach, medium timeline
    \item Aggressive: Maximum change, shorter timeline
\end{itemize}
The key is choosing a scenario that aligns with organizational risk tolerance while meeting target deadlines.
\end{explanation}

\begin{observation}
Standard scenarios for team-based model:
\begin{itemize}
    \item Conservative:
        \begin{itemize}
            \item Team Reduction: 10\%
            \item Service Efficiency: +15\%
            \item Operational Overhead: -5\%
            \item Monthly Savings: MAI/24
        \end{itemize}
    \item Moderate:
        \begin{itemize}
            \item Team Reduction: 15\%
            \item Service Efficiency: +25\%
            \item Operational Overhead: -10\%
            \item Monthly Savings: MAI/18
        \end{itemize}
    \item Aggressive:
        \begin{itemize}
            \item Team Reduction: 20\%
            \item Service Efficiency: +35\%
            \item Operational Overhead: -15\%
            \item Monthly Savings: MAI/12
        \end{itemize}
\end{itemize}
\end{observation}

\begin{explanation}
Each scenario component is carefully balanced:
\begin{itemize}
    \item Team Reduction correlates with automation/efficiency gains
    \item Service Efficiency improvements offset reduced team size
    \item Operational Overhead reduction ensures sustainable operations
    \item Monthly Savings determine break-even timeline
\end{itemize}
This ensures that efficiency gains can realistically support team size changes while maintaining service levels.
\end{explanation}

\subsection{Ticket-Based Scenarios}

\begin{explanation}
Ticket-based scenarios focus on process optimization rather than organizational changes:
\begin{itemize}
    \item Processing Time improvements through automation and streamlining
    \item Resource Allocation optimization through better routing and assignment
    \item Quality Impact through standardization and validation
\end{itemize}
Each scenario balances these elements differently to achieve the target outcome.
\end{explanation}

\begin{observation}
Standard scenarios for ticket-based model:
\begin{itemize}
    \item Conservative:
        \begin{itemize}
            \item Processing Time: -10\%
            \item Resource Allocation: -15\%
            \item Quality Impact: +5\%
            \item Monthly Savings: MAI/24
        \end{itemize}
    \item Moderate:
        \begin{itemize}
            \item Processing Time: -20\%
            \item Resource Allocation: -25\%
            \item Quality Impact: +10\%
            \item Monthly Savings: MAI/18
        \end{itemize}
    \item Aggressive:
        \begin{itemize}
            \item Processing Time: -30\%
            \item Resource Allocation: -35\%
            \item Quality Impact: +15\%
            \item Monthly Savings: MAI/12
        \end{itemize}
\end{itemize}
\end{observation}

\begin{explanation}
The ticket-based scenarios are designed to:
\begin{itemize}
    \item Achieve efficiency gains through process improvements
    \item Maintain or improve quality while reducing resources
    \item Balance speed of implementation with organizational capacity
    \item Ensure sustainable improvements through measured changes
\end{itemize}
This approach focuses on systematic process improvement rather than organizational restructuring.
\end{explanation}

\section{Conclusion}

\begin{explanation}
This technical documentation provides a framework for:
\begin{itemize}
    \item Understanding team dependencies and their impact on efficiency
    \item Analyzing distribution models and cost structures
    \item Applying target-based planning to simplify platform solutions
    \item Choosing between team-based and ticket-based approaches
\end{itemize}

The methodology emphasizes:
\begin{itemize}
    \item Working backwards from clear business targets
    \item Minimizing complexity in solution design
    \item Balancing efficiency gains with operational stability
    \item Adapting approaches based on organizational context
\end{itemize}
\end{explanation}

\end{document} 